% Options for packages loaded elsewhere
\PassOptionsToPackage{unicode}{hyperref}
\PassOptionsToPackage{hyphens}{url}
%
\documentclass[
]{article}
\usepackage{amsmath,amssymb}
\usepackage{iftex}
\ifPDFTeX
  \usepackage[T1]{fontenc}
  \usepackage[utf8]{inputenc}
  \usepackage{textcomp} % provide euro and other symbols
\else % if luatex or xetex
  \usepackage{unicode-math} % this also loads fontspec
  \defaultfontfeatures{Scale=MatchLowercase}
  \defaultfontfeatures[\rmfamily]{Ligatures=TeX,Scale=1}
\fi
\usepackage{lmodern}
\ifPDFTeX\else
  % xetex/luatex font selection
\fi
% Use upquote if available, for straight quotes in verbatim environments
\IfFileExists{upquote.sty}{\usepackage{upquote}}{}
\IfFileExists{microtype.sty}{% use microtype if available
  \usepackage[]{microtype}
  \UseMicrotypeSet[protrusion]{basicmath} % disable protrusion for tt fonts
}{}
\makeatletter
\@ifundefined{KOMAClassName}{% if non-KOMA class
  \IfFileExists{parskip.sty}{%
    \usepackage{parskip}
  }{% else
    \setlength{\parindent}{0pt}
    \setlength{\parskip}{6pt plus 2pt minus 1pt}}
}{% if KOMA class
  \KOMAoptions{parskip=half}}
\makeatother
\usepackage{xcolor}
\usepackage[margin=1in]{geometry}
\usepackage{color}
\usepackage{fancyvrb}
\newcommand{\VerbBar}{|}
\newcommand{\VERB}{\Verb[commandchars=\\\{\}]}
\DefineVerbatimEnvironment{Highlighting}{Verbatim}{commandchars=\\\{\}}
% Add ',fontsize=\small' for more characters per line
\usepackage{framed}
\definecolor{shadecolor}{RGB}{248,248,248}
\newenvironment{Shaded}{\begin{snugshade}}{\end{snugshade}}
\newcommand{\AlertTok}[1]{\textcolor[rgb]{0.94,0.16,0.16}{#1}}
\newcommand{\AnnotationTok}[1]{\textcolor[rgb]{0.56,0.35,0.01}{\textbf{\textit{#1}}}}
\newcommand{\AttributeTok}[1]{\textcolor[rgb]{0.13,0.29,0.53}{#1}}
\newcommand{\BaseNTok}[1]{\textcolor[rgb]{0.00,0.00,0.81}{#1}}
\newcommand{\BuiltInTok}[1]{#1}
\newcommand{\CharTok}[1]{\textcolor[rgb]{0.31,0.60,0.02}{#1}}
\newcommand{\CommentTok}[1]{\textcolor[rgb]{0.56,0.35,0.01}{\textit{#1}}}
\newcommand{\CommentVarTok}[1]{\textcolor[rgb]{0.56,0.35,0.01}{\textbf{\textit{#1}}}}
\newcommand{\ConstantTok}[1]{\textcolor[rgb]{0.56,0.35,0.01}{#1}}
\newcommand{\ControlFlowTok}[1]{\textcolor[rgb]{0.13,0.29,0.53}{\textbf{#1}}}
\newcommand{\DataTypeTok}[1]{\textcolor[rgb]{0.13,0.29,0.53}{#1}}
\newcommand{\DecValTok}[1]{\textcolor[rgb]{0.00,0.00,0.81}{#1}}
\newcommand{\DocumentationTok}[1]{\textcolor[rgb]{0.56,0.35,0.01}{\textbf{\textit{#1}}}}
\newcommand{\ErrorTok}[1]{\textcolor[rgb]{0.64,0.00,0.00}{\textbf{#1}}}
\newcommand{\ExtensionTok}[1]{#1}
\newcommand{\FloatTok}[1]{\textcolor[rgb]{0.00,0.00,0.81}{#1}}
\newcommand{\FunctionTok}[1]{\textcolor[rgb]{0.13,0.29,0.53}{\textbf{#1}}}
\newcommand{\ImportTok}[1]{#1}
\newcommand{\InformationTok}[1]{\textcolor[rgb]{0.56,0.35,0.01}{\textbf{\textit{#1}}}}
\newcommand{\KeywordTok}[1]{\textcolor[rgb]{0.13,0.29,0.53}{\textbf{#1}}}
\newcommand{\NormalTok}[1]{#1}
\newcommand{\OperatorTok}[1]{\textcolor[rgb]{0.81,0.36,0.00}{\textbf{#1}}}
\newcommand{\OtherTok}[1]{\textcolor[rgb]{0.56,0.35,0.01}{#1}}
\newcommand{\PreprocessorTok}[1]{\textcolor[rgb]{0.56,0.35,0.01}{\textit{#1}}}
\newcommand{\RegionMarkerTok}[1]{#1}
\newcommand{\SpecialCharTok}[1]{\textcolor[rgb]{0.81,0.36,0.00}{\textbf{#1}}}
\newcommand{\SpecialStringTok}[1]{\textcolor[rgb]{0.31,0.60,0.02}{#1}}
\newcommand{\StringTok}[1]{\textcolor[rgb]{0.31,0.60,0.02}{#1}}
\newcommand{\VariableTok}[1]{\textcolor[rgb]{0.00,0.00,0.00}{#1}}
\newcommand{\VerbatimStringTok}[1]{\textcolor[rgb]{0.31,0.60,0.02}{#1}}
\newcommand{\WarningTok}[1]{\textcolor[rgb]{0.56,0.35,0.01}{\textbf{\textit{#1}}}}
\usepackage{graphicx}
\makeatletter
\def\maxwidth{\ifdim\Gin@nat@width>\linewidth\linewidth\else\Gin@nat@width\fi}
\def\maxheight{\ifdim\Gin@nat@height>\textheight\textheight\else\Gin@nat@height\fi}
\makeatother
% Scale images if necessary, so that they will not overflow the page
% margins by default, and it is still possible to overwrite the defaults
% using explicit options in \includegraphics[width, height, ...]{}
\setkeys{Gin}{width=\maxwidth,height=\maxheight,keepaspectratio}
% Set default figure placement to htbp
\makeatletter
\def\fps@figure{htbp}
\makeatother
\setlength{\emergencystretch}{3em} % prevent overfull lines
\providecommand{\tightlist}{%
  \setlength{\itemsep}{0pt}\setlength{\parskip}{0pt}}
\setcounter{secnumdepth}{-\maxdimen} % remove section numbering
\ifLuaTeX
  \usepackage{selnolig}  % disable illegal ligatures
\fi
\IfFileExists{bookmark.sty}{\usepackage{bookmark}}{\usepackage{hyperref}}
\IfFileExists{xurl.sty}{\usepackage{xurl}}{} % add URL line breaks if available
\urlstyle{same}
\hypersetup{
  pdftitle={Challenge-5},
  pdfauthor={CHENJUNYURYAN},
  hidelinks,
  pdfcreator={LaTeX via pandoc}}

\title{Challenge-5}
\author{CHENJUNYURYAN}
\date{2023-09-13}

\begin{document}
\maketitle

\hypertarget{questions}{%
\subsection{Questions}\label{questions}}

\hypertarget{question-1-local-variable-shadowing}{%
\paragraph{Question-1: Local Variable
Shadowing}\label{question-1-local-variable-shadowing}}

Create an R function that defines a global variable called \texttt{x}
with a value of 5. Inside the function, declare a local variable also
named \texttt{x} with a value of 10. Print the value of \texttt{x} both
inside and outside the function to demonstrate shadowing.

\textbf{Solutions:}

\begin{Shaded}
\begin{Highlighting}[]
\NormalTok{x }\OtherTok{\textless{}{-}} \DecValTok{5} \CommentTok{\#global variable that is define outside of function}
\NormalTok{shadowing\_function }\OtherTok{\textless{}{-}} \ControlFlowTok{function}\NormalTok{(x) \{x }\OtherTok{\textless{}{-}} \DecValTok{10}\NormalTok{\} }\CommentTok{\#local variable that is define within function}
\FunctionTok{print}\NormalTok{(x)}
\end{Highlighting}
\end{Shaded}

\begin{verbatim}
## [1] 5
\end{verbatim}

\begin{Shaded}
\begin{Highlighting}[]
\FunctionTok{print}\NormalTok{(}\FunctionTok{shadowing\_function}\NormalTok{(x))}
\end{Highlighting}
\end{Shaded}

\begin{verbatim}
## [1] 10
\end{verbatim}

\begin{Shaded}
\begin{Highlighting}[]
\CommentTok{\# variables initialise inside the function will not be available outside the function}
\CommentTok{\# sprintf }
\end{Highlighting}
\end{Shaded}

\hypertarget{question-2-modify-global-variable}{%
\paragraph{Question-2: Modify Global
Variable}\label{question-2-modify-global-variable}}

Create an R function that takes an argument and adds it to a global
variable called \texttt{total}. Call the function multiple times with
different arguments to accumulate the values in \texttt{total}.

\textbf{Solutions:}

\begin{Shaded}
\begin{Highlighting}[]
\NormalTok{total }\OtherTok{\textless{}{-}} \DecValTok{0} \CommentTok{\# global variable that is define outside of function}
\NormalTok{add\_to\_total }\OtherTok{\textless{}{-}} \ControlFlowTok{function}\NormalTok{(x)\{ total }\OtherTok{\textless{}\textless{}{-}}\NormalTok{ total }\SpecialCharTok{+}\NormalTok{ x\}}
\FunctionTok{add\_to\_total}\NormalTok{(}\DecValTok{5}\NormalTok{) }\CommentTok{\#call function with argument 5 (after inputting this function does (global) total \textless{}{-} 5 (0+5))}

\FunctionTok{print}\NormalTok{(total) }\CommentTok{\# total \textless{}{-} (0+5)}
\end{Highlighting}
\end{Shaded}

\begin{verbatim}
## [1] 5
\end{verbatim}

\begin{Shaded}
\begin{Highlighting}[]
\FunctionTok{add\_to\_total}\NormalTok{(}\DecValTok{10}\NormalTok{) }\CommentTok{\#call function with argument 10 (after inputting this function is the (global) total \textless{}{-} 15 (5+10) )}

\FunctionTok{print}\NormalTok{(total) }\CommentTok{\# using last total which was (0+5) hence total \textless{}{-} ((0+5)+10)}
\end{Highlighting}
\end{Shaded}

\begin{verbatim}
## [1] 15
\end{verbatim}

\begin{Shaded}
\begin{Highlighting}[]
\FunctionTok{add\_to\_total}\NormalTok{(}\DecValTok{15}\NormalTok{) }\CommentTok{\#argument 3 is 15, call function with argument 15}

\FunctionTok{print}\NormalTok{(total) }
\end{Highlighting}
\end{Shaded}

\begin{verbatim}
## [1] 30
\end{verbatim}

\begin{Shaded}
\begin{Highlighting}[]
\CommentTok{\# (\textless{}\textless{}{-}) creates a global variable inside a function}
\end{Highlighting}
\end{Shaded}

\hypertarget{question-3-global-and-local-interaction}{%
\paragraph{Question-3: Global and Local
Interaction}\label{question-3-global-and-local-interaction}}

Write an R program that includes a global variable \texttt{total} with
an initial value of 100. Create a function that takes an argument, adds
it to \texttt{total}, and returns the updated \texttt{total}.
Demonstrate how this function interacts with the global variable.

\textbf{Solutions:}

\begin{Shaded}
\begin{Highlighting}[]
\NormalTok{total }\OtherTok{\textless{}{-}} \DecValTok{100} \CommentTok{\#global variable}
\NormalTok{updated\_total }\OtherTok{\textless{}{-}} \ControlFlowTok{function}\NormalTok{(x)\{total }\OtherTok{\textless{}{-}}\NormalTok{ total}\SpecialCharTok{+}\NormalTok{x }
\FunctionTok{return}\NormalTok{(total)\} }\CommentTok{\#dont understand the return part help ! so it }


\FunctionTok{print}\NormalTok{(}\FunctionTok{updated\_total}\NormalTok{(}\DecValTok{50}\NormalTok{))}
\end{Highlighting}
\end{Shaded}

\begin{verbatim}
## [1] 150
\end{verbatim}

\begin{Shaded}
\begin{Highlighting}[]
\FunctionTok{print}\NormalTok{(total)}
\end{Highlighting}
\end{Shaded}

\begin{verbatim}
## [1] 100
\end{verbatim}

\begin{Shaded}
\begin{Highlighting}[]
\CommentTok{\#return, returns the last output of a function atuomatically }
\end{Highlighting}
\end{Shaded}

\hypertarget{question-4-nested-functions}{%
\paragraph{Question-4: Nested
Functions}\label{question-4-nested-functions}}

Define a function \texttt{outer\_function} that declares a local
variable \texttt{x} with a value of 5. Inside \texttt{outer\_function},
define another function \texttt{inner\_function} that prints the value
of \texttt{x}. Call both functions to show how the inner function
accesses the variable from the outer function's scope.

\textbf{Solutions:}

\begin{Shaded}
\begin{Highlighting}[]
\NormalTok{outer\_function }\OtherTok{\textless{}{-}} \ControlFlowTok{function}\NormalTok{() \{x }\OtherTok{\textless{}{-}} \DecValTok{5}
\NormalTok{inner\_function }\OtherTok{\textless{}{-}} \ControlFlowTok{function}\NormalTok{()\{}\FunctionTok{print}\NormalTok{(x)\} }
\FunctionTok{inner\_function}\NormalTok{() \}  }\CommentTok{\# local variable }

\CommentTok{\#what is the difference between function() and function(x), function() is fixed whereas function(x)}
\FunctionTok{print}\NormalTok{(}\FunctionTok{outer\_function}\NormalTok{())}
\end{Highlighting}
\end{Shaded}

\begin{verbatim}
## [1] 5
## [1] 5
\end{verbatim}

\begin{Shaded}
\begin{Highlighting}[]
\NormalTok{?return}
\end{Highlighting}
\end{Shaded}

\begin{verbatim}
## starting httpd help server ... done
\end{verbatim}

\hypertarget{question-5-meme-generator-function}{%
\paragraph{Question-5: Meme Generator
Function}\label{question-5-meme-generator-function}}

Create a function that takes a text input and generates a humorous meme
with the text overlaid on an image of your choice. You can use the
\texttt{magick} package for image manipulation. You can find more
details about the commands offered by the package, with some examples of
annotating images here:
\url{https://cran.r-project.org/web/packages/magick/vignettes/intro.html}

\textbf{Solutions:}

\begin{Shaded}
\begin{Highlighting}[]
\FunctionTok{library}\NormalTok{(magick)}
\end{Highlighting}
\end{Shaded}

\begin{verbatim}
## Linking to ImageMagick 6.9.12.93
## Enabled features: cairo, freetype, fftw, ghostscript, heic, lcms, pango, raw, rsvg, webp
## Disabled features: fontconfig, x11
\end{verbatim}

\begin{Shaded}
\begin{Highlighting}[]
\NormalTok{create\_meme }\OtherTok{\textless{}{-}} \ControlFlowTok{function}\NormalTok{(text,image\_path,output\_path) \{}
\NormalTok{  meme }\OtherTok{\textless{}{-}} \FunctionTok{image\_read}\NormalTok{(image\_path)}
\NormalTok{  meme }\OtherTok{\textless{}{-}} \FunctionTok{image\_annotate}\NormalTok{(meme,text,}\AttributeTok{gravity=}\StringTok{"center"}\NormalTok{,}\AttributeTok{size =} \DecValTok{30}\NormalTok{, }\AttributeTok{color =} \StringTok{"black"}\NormalTok{)}
  \FunctionTok{image\_write}\NormalTok{(meme, }\AttributeTok{path =}\NormalTok{ output\_path)}
\NormalTok{\}}

\FunctionTok{create\_meme}\NormalTok{(}\StringTok{"A potato flew around my room"}\NormalTok{,}\StringTok{"POTATO.jpg"}\NormalTok{,}\StringTok{"output\_meme.jpg"}\NormalTok{)}
\end{Highlighting}
\end{Shaded}

\hypertarget{question-6-text-analysis-game}{%
\paragraph{Question-6: Text Analysis
Game}\label{question-6-text-analysis-game}}

Develop a text analysis game in which the user inputs a sentence, and
the R function provides statistics like the number of words, characters,
and average word length. Reward the user with a ``communication skill
level'' based on their input.

\textbf{Solutions:}

\begin{Shaded}
\begin{Highlighting}[]
\NormalTok{text\_analysis\_game }\OtherTok{\textless{}{-}} \ControlFlowTok{function}\NormalTok{() \{}
  \FunctionTok{cat}\NormalTok{(}\StringTok{"Welcome to the Text Analysis Game!}\SpecialCharTok{\textbackslash{}n}\StringTok{"}\NormalTok{)}
  
  \CommentTok{\# Prompt the user for input}
\NormalTok{  user\_input }\OtherTok{\textless{}{-}} \FunctionTok{readline}\NormalTok{(}\AttributeTok{prompt =} \StringTok{"Enter a sentence: "}\NormalTok{)}
  
  \CommentTok{\# Tokenize the input sentence into words}
\NormalTok{  words }\OtherTok{\textless{}{-}} \FunctionTok{unlist}\NormalTok{(}\FunctionTok{strsplit}\NormalTok{(user\_input, }\StringTok{"}\SpecialCharTok{\textbackslash{}\textbackslash{}}\StringTok{s+"}\NormalTok{))}
  \CommentTok{\# Calculate statistics}
\NormalTok{  num\_words }\OtherTok{\textless{}{-}} \FunctionTok{length}\NormalTok{(words)}
\NormalTok{  num\_characters }\OtherTok{\textless{}{-}} \FunctionTok{nchar}\NormalTok{(user\_input)}
\NormalTok{  average\_word\_length }\OtherTok{\textless{}{-}} \FunctionTok{mean}\NormalTok{(}\FunctionTok{nchar}\NormalTok{(words))}
  
  \CommentTok{\# Determine communication skill level}
\NormalTok{  skill\_level }\OtherTok{\textless{}{-}} \ControlFlowTok{if}\NormalTok{ (average\_word\_length }\SpecialCharTok{\textgreater{}=} \DecValTok{7}\NormalTok{) \{}
    \StringTok{"Excellent"}
\NormalTok{  \} }\ControlFlowTok{else} \ControlFlowTok{if}\NormalTok{ (average\_word\_length }\SpecialCharTok{\textgreater{}=} \DecValTok{5}\NormalTok{) \{}
    \StringTok{"Good"}
\NormalTok{  \} }\ControlFlowTok{else}\NormalTok{ \{}
    \StringTok{"Needs Improvement"}
\NormalTok{  \}}
  
  \CommentTok{\# Display the statistics and reward}
  \FunctionTok{cat}\NormalTok{(}\StringTok{"}\SpecialCharTok{\textbackslash{}n}\StringTok{{-}{-}{-} Text Analysis Results {-}{-}{-}}\SpecialCharTok{\textbackslash{}n}\StringTok{"}\NormalTok{)}
  \FunctionTok{cat}\NormalTok{(}\StringTok{"Number of words: "}\NormalTok{, num\_words, }\StringTok{"}\SpecialCharTok{\textbackslash{}n}\StringTok{"}\NormalTok{)}
  \FunctionTok{cat}\NormalTok{(}\StringTok{"Number of characters: "}\NormalTok{, num\_characters, }\StringTok{"}\SpecialCharTok{\textbackslash{}n}\StringTok{"}\NormalTok{)}
  \FunctionTok{cat}\NormalTok{(}\StringTok{"Average word length: "}\NormalTok{, }\FunctionTok{round}\NormalTok{(average\_word\_length, }\DecValTok{2}\NormalTok{), }\StringTok{"}\SpecialCharTok{\textbackslash{}n}\StringTok{"}\NormalTok{)}
  \FunctionTok{cat}\NormalTok{(}\StringTok{"Communication skill level: "}\NormalTok{, skill\_level, }\StringTok{"}\SpecialCharTok{\textbackslash{}n}\StringTok{"}\NormalTok{)}
  
  \CommentTok{\# Provide feedback based on skill level}
  \FunctionTok{cat}\NormalTok{(}\StringTok{"}\SpecialCharTok{\textbackslash{}n}\StringTok{Congratulations! Your communication skills are "}\NormalTok{, skill\_level, }\StringTok{".}\SpecialCharTok{\textbackslash{}n}\StringTok{"}\NormalTok{)}
\NormalTok{\}}

\CommentTok{\# Run the game}
\end{Highlighting}
\end{Shaded}


\end{document}
